\documentclass{monografiaime}
\usepackage{wrapfig,lipsum,booktabs}
\usepackage{makecell}
\usepackage{amsmath}
\DeclareMathOperator*{\argmax}{argmax}
%\documentclass[quali]{monografiaime}

\nome{Vítor Kei Taira Tamada}
\titulo{Deep Q Learning para ensinar inteligência artificial a jogar Asteroids}
\mestrado{Ciência da Computação}
\orientador{Prof. Dr. Denis Deratani Mauá}{IME-USP}

\datadeposito{novembro de 2018} % data do deposito

% Cria o comando \argmax{}
\DeclareMathOperator*{\argmax}{arg\,max}

% Caso sua defesa já tenha ocorrido e esta é a versão corrigida do seu trabalho,
% insira a data de defesa e o nome dos professores que participaram da banca nas
% linhas abaixo e as descomente.
%\dadosdefesa{01/01/1900}%
%            {\orientadorcominstituto}%
%            {Prof. Dr. Dois - IME-USP}%
%            {Prof. Dr. Três - UNICAMP}


% ---------------------------------------------------------------------------- %
% ---------------------------------------------------------------------------- %
% PARTE PADRÃO
%
% Não precisa mexer daqui para baixo
% ---------------------------------------------------------------------------- %
% ---------------------------------------------------------------------------- %
\begin{document}
\formatacaopaginaspreliminares


% ---------------------------------------------------------------------------- %
% CAPA
% ---------------------------------------------------------------------------- %
\capas

\pagenumbering{roman}        % Começamos a numeração romana.
\input chapters/resumoagradecimento


% ---------------------------------------------------------------------------- %
% Sumário
% ---------------------------------------------------------------------------- %
\tableofcontents             % Imprime o sumário.


% ---------------------------------------------------------------------------- %
% Corpo do Trabalho
% ---------------------------------------------------------------------------- %
\formatacaocorpotrabalho
%\formatacaocorpotrabalho[\onehalfspacing]

\input capitulos             % Insere conteúdo do arquivo: 'capitulos.tex'

% Cabeçalho para os apêndices.
\formatacaoapendices
\appendix
\input apendices             % Insere conteúdo do arquivo: 'apendices.tex'


% ---------------------------------------------------------------------------- %
% Bibliografia
% ---------------------------------------------------------------------------- %
\backmatter \singlespacing   % Espaçamento simples.
\bibliographystyle{alpha-ime}% Citação bibliográfica alpha.
\bibliography{bibliografia}  % Insere conteúdo do arquivo: 'bibliografia.bib'


% ---------------------------------------------------------------------------- %
% Índice remissivo
% ---------------------------------------------------------------------------- %
\printindex                  % Imprime o índice remissivo no documento.

\end{document}
