%%% ---------------------------------------------------------------------------- %
%\chapter*{Agradecimentos}
%%% ---------------------------------------------------------------------------- %
%Primeiro, gostaria de agradecer meus pais, Getúlio e Edna, e meu irmão, Dan, por todo apoio que me deram nas épocas e decisões mais difíceis, e paciência que tiveram com meus erros não só ao longo de minha graduação, como de toda minha vida.
%Tudo isso certamente fez uma grande diferença para chegar onde estou e tornar quem eu sou hoje.
%\\
%
%Um agradecimento especial aos meus amigos Fabio, Matheus, Leo, e Thomas.
%Podemos ter nos encontrado cada vez menos nos últimos anos, mas as trocas de mensagens, as jogatinas pela internet, as vezes que conseguimos nos encontrar e as viagens foram sempre muito bem aproveitadas, sempre ajudando a levantar meu ânimo e seguir em frente.
%Ter pessoas com quem contar e confiar a qualquer momento, me sentir querido na vida de alguém sempre colocou um sorriso em meu rosto nas horas mais difíceis.
%Conhecê-los foi, sem dúvidas, a melhor coisa que aconteceu em minha vida.
%\\
%
%Muito obrigado a todos meus amigos do IME com os quais compartilhei muito aprendizado e experiências ao longo desses anos.
%Ter alguém para pedir ajuda, estudar junto e fazer trabalhos, assim como descontrair e relaxar foi essencial na graduação.
%\\
%
%Expresso, em particular, minha gratidão ao meu orientador, por não deixar que eu finalizasse meu trabalho com um resultado muito abaixo do esperado, e ao meu amigo do IME Fábio Tanaka, por me ajudar e ouvir nos dias de desespero quando o projeto não andava para frente, mesmo tendo seu próprio trabalho de conclusão de curso para se preocupar e estando do outro lado do mundo, fazendo nossas conversas serem muitas vezes durante a madrugada para ele.
%\\
%
%E a todos os amigos de fora do IME, professores e colegas que, direta ou indiretamente, contribuíram para meu aprendizado e formação neste instituto. 

%% ---------------------------------------------------------------------------- %
\chapter*{Resumo}
%% ---------------------------------------------------------------------------- %
%
\noindent%
TAMADA, V. K. T. \textbf{Estudo de caso de Deep Q-Learning}. Trabalho de Conclusão de Curso
%Deep Q-Learning para ensinar inteligência artificial a jogar Asteroids
%FONSECA, R. L. \textbf{PsyChO: The Ball}. Trabalho de Conclusão de Curso
 - Instituto de Matemática e Estatística, Universidade de São Paulo,
São Paulo, 2018.
\\

Visualização de imagem, abstração de informação e aprendizado por recompensa são tarefas que seres humanos aprendem consideravelmente rápido.
Computadores, por outro lado, podem levar horas ou até dias para aprender algo que pessoas fariam em segundos, principalmente quando envolve interpretação de imagens.
Utilizando aprendizado profundo em conjunto de aprendizado por reforço, este trabalho busca estudar a eficiência de \textit{deep Q-learning} para o aprendizado de um agente que recebe apenas a tela de ambientes como entrada em três ambientes diferentes.
%Jogos eletrônicos se tornaram comuns na vida de muitas pessoas nos últimos anos, seja em consoles de mesa tradicionais, em portáteis, ou em celulares.
%Eles normalmente são simples e intuitivos, para qualquer um poder começar a qualquer momento e aprender rapidamente como se joga.
%Porém, isso não é uma tarefa tão fácil para computadores.
%Utilizando \textit{deep learning} em conjunto com aprendizado por reforço, o objetivo deste trabalho é produzir uma inteligência artificial que aprenda a jogar o jogo de Atari2600 \textit{Asteroids}.
\\

\noindent%
\textbf{Palavras-chave:} inteligência artificial, deep q-learning, estudo de caso

%
%Desenvolvimento de jogos é uma área da computação repleta de desafios. O jogo digital PsyChO: The Ball, um \textit{Top Down Shooter} psicodélico minimalista, foi produzido, desde seu início, utilizando apenas software livre. Feito no arcabouço \textit{LÖVE} com a linguagem Lua, o processo de desenvolvimento passou por todas as etapas necessárias para a produção de um jogo. Ele foi inspirado no Game Design dos jogos digitais \textit{Hotline Miami}, \textit{Touhou} e \textit{Hexagon} e, entre suas melhores características, temos sua jogabilidade frenética e efeitos especiais \textit{juicy}.
%\\
%
%\noindent%
%\textbf{Palavras-chave:} desenvolvimento de jogos, game design, juiciness, top down shooter.
