% labels:
% cap:conclusoes

% ---------------------------------------------------------------------------- %
\chapter{Conclusão}
\label{cap:conclusoes}
% ---------------------------------------------------------------------------- %
Este trabalho permitiu conhecer e explorar uma técnica de aprendizado de máquina que não é discutida nas disciplinas da graduação.
A sua capacidade de resolver problemas complexos mostrou-se compensada pela dificuldade de se utilizar com sucesso.

As observações feitas refletiram as expectativas mesmo que apenas em parte.
Todos utilizaram uma matriz como entrada, mas graus diferentes de dificuldade para se aprender, como o aumento do tamanho da entrada, do espaço de estados e das ações disponíveis.
O ambiente mais simples, com poucos estados e com recompensa e penalidades bem definidas obteve o maior grau de sucesso;
o de média complexidade apresentou resultados promissores, mas sem sucessos consistentes como no anterior;
e o mais complexo, contrariando as expectativas iniciais, foi pouco promissor.

%Este trabalho permitiu explorar uma técnica de aprendizagem de máquina que não é discutida nas disciplinas da graduação, ainda que seja uma junção de duas que são abordadas.
%A complexidade dos problemas que ela é capaz de resolver foi balanceada pela dificuldade de se utilizar com sucesso no ambiente almejado por este projeto.
%Há muitos hiper-parâmetros para se ajustar e o treinamento leva horas, até mesmo dias, para terminar.
%Além disso, não existem regras e teorias bem definidas para quais valores devem ser adotados, apenas relatos de casos bem sucedidos e algumas direções baseadas neles de quais podem ser bons.
%Mesmo assim, seu estudo e desenvolvimento foram interessantes, ainda que sucesso com o \textit{Asteroids} não tenha sido obtido no final por conta de sua complexidade.



%A escolha do tema não foi muito difícil. Já estava seguindo a trilha de inteligência artificial do curso de ciência da computação e o grande interesse por jogos mostrou rapidamente uma aplicação do que foi estudado nessa área.
%A implementação também não foi muito problemática, uma vez que o estudo e desenvolvimento de redes neurais profundas e respectivas aplicações para detecção de imagem tiveram amplo destaque nos últimos anos.
%Isso fez muitos guias, escritos ou em vídeo, surgirem pela internet, assim como dúvidas dos mais variados tipos em fóruns do assunto, que serviram de grande ajuda para escrever o código.

%Entretanto, o maior obstáculo teve que ser solucionado por conta própria: encontrar os hiper-parâmetros e funções certos para que o agente aprendesse.
%Existem muitos hiper-parâmetros para ajustar e o treinamento leva horas, podendo até mesmo passar de um dia para o outro ou até além disso para terminar.
%Também não existem regras e teorias bem definidas para a escolha de muitos deles, apenas relatos de alguns que foram usados e funcionaram, o que dá uma noção de valores bons no máximo.
%Os testes com ambientes mais simples serviram de grande ajuda ao garantir e demonstrar o funcionamento da técnica de aprendizado \textit{deep Q-learning}, indicando que a dificuldade estava na escolha correta dos hiper-parâmetros pelo fato de \textit{Asteroids} ser um ambiente muito mais complexo.

%///////////%


%A escolha do tema foi relativamente fácil, pois ao longo da graduação fui cada vez me envolvendo mais na área de desenvolvimento de jogos. \textit{PsyChO: The Ball} foi provavelmente o projeto no qual mais dediquei tempo e paixão durante o curso, tanto pelo objetivo de servir como trabalho final de formatura, quanto por ser uma adaptação do meu primeiro jogo feito no início da graduação. Nele, pude ver claramente o progresso que tive como programador, \textit{game designer} e até mesmo artista.

%Todos os desafios e frustrações que foram superados durante esse processo levaram a um resultado muito gratificante. Ter a chance de apresentar esse jogo para amigos e membros da Universidade de São Paulo durante os eventos expositivos da UspGameDev faz valer a pena todo o esforço pesquisando em fóruns e artigos como \textit{debuggar} um problema ou como implementar um algoritmo.

%É com grande orgulho que levo esse projeto comigo ao fim da graduação, servindo de portofólio do que eu consigo realizar como um desenvolvedor de jogos.

% ---------------------------------------------------------------------------- %
%\section{Graduação e o Trabalho de Formatura}
%\label{sec:materias_utilizadas}

%As matérias que me ensinaram técnicas e conceitos de programação: \textbf{Introdução à Ciência da Computação}; \textbf{Algoritmos e Estruturas de Dados I}; \textbf{Técnicas de Programação I}; \textbf{Algoritmos e Estruturas de Dados II} e \textbf{Conceitos Fundamentais de Linguagens de Programação}, foram todas essenciais para formar a base de conhecimento que tive em programação.

%As matérias \textbf{Análise de Algoritmos} e \textbf{Introdução à Lógica e Verificação de Programas} me deram uma maior formalidade em análisar conceitos lógicos e algoritmos, me fazendo entender melhor otimização e estruturação de código. Foram de imensa ajuda para a produção de \textit{PsyChO: The Ball}.

%As matérias \textbf{Programação Concorrente} e \textbf{Sistemas Operacionais} me deram o conhecimento de paralelismo e corotinas, o qual apliquei diretamente no sistema de leitura de \textit{scripts} pra níveis no \textbf{PsyChO: The Ball}.

%As matérias: \textbf{Atividade Curricular em Cultura e Extensão}; \textbf{Design e Programação de Games}; \textbf{Laboratório de Programação II}; \textbf{Laboratório de Programação Extrema} e \textbf{Computação Móvel}, ofereceram atividades nas quais tive de fazer, na prática, um jogo (seja esse digital ou analógico). Em cada uma aprendi com os sucessos (e com os vários erros), podendo assim desenvolver e aprimorar técnicas de desenvolvimento de jogos. Quero fazer um agradecimento especial aos professores que me forneceram essas oportunidades de aplicar conhecimentos computacionais, em programação de jogos, algo que me incentivou a aprender e descobrir coisas novas.

%Por último, o grupo extracurricular \textbf{UspGameDev} me ensinou muito mais do que eu poderia imaginar sobre desenvolvimento de jogos, \textit{game design}, \textit{game programming patterns} e me guiou durante todos projetos lúdicos extra-acadêmicos.

% ---------------------------------------------------------------------------- %
%\section{Trabalhos Futuros}
%\label{sec:trabalhos_futuros}

%Meu objetivo é continuar trabalhando em \textit{PsyChO: The Ball} até chegar no estado desejado: 5 níveis, cada um com mecânicas e inimigos únicos e interessantes. É desejado lançar o jogo assim que possível em alguma plataforma digital de distribuição de jogos como a \textit{Steam}\footnote{store.steampowered.com}, \textit{Humble Bundle}\footnote{https://www.humblebundle.com/} ou \textit{Itch.io}\footnote{https://itch.io/}.

%Além disso, quero continuar meus estudos nas área de Game Design e desenvolvimento de jogos, buscando sempre aprender novas técnicas, descobrir novas ferramentas e superar os diversos desafios que compõe essa área.
