% ---------------------------------------------------------------------------- %
\chapter{Conclusões - Parte Subjetiva}
\label{cap:conclusoes}
% ---------------------------------------------------------------------------- %

A escolha do tema foi relativamente fácil, pois ao longo da graduação fui cada vez me envolvendo mais na área de desenvolvimento de jogos. \textit{PsyChO: The Ball} foi provavelmente o projeto no qual mais dediquei tempo e paixão durante o curso, tanto pelo objetivo de servir como trabalho final de formatura, quanto por ser uma adaptação do meu primeiro jogo feito no início da graduação. Nele, pude ver claramente o progresso que tive como programador, \textit{game designer} e até mesmo artista.

Todos os desafios e frustrações que foram superados durante esse processo levaram a um resultado muito gratificante. Ter a chance de apresentar esse jogo para amigos e membros da Universidade de São Paulo durante os eventos expositivos da UspGameDev faz valer a pena todo o esforço pesquisando em fóruns e artigos como \textit{debuggar} um problema ou como implementar um algoritmo.

É com grande orgulho que levo esse projeto comigo ao fim da graduação, servindo de portofólio do que eu consigo realizar como um desenvolvedor de jogos.

% ---------------------------------------------------------------------------- %
\section{Graduação e o Trabalho de Formatura}
\label{sec:materias_utilizadas}

As matérias que me ensinaram técnicas e conceitos de programação: \textbf{Introdução à Ciência da Computação}; \textbf{Algoritmos e Estruturas de Dados I}; \textbf{Técnicas de Programação I}; \textbf{Algoritmos e Estruturas de Dados II} e \textbf{Conceitos Fundamentais de Linguagens de Programação}, foram todas essenciais para formar a base de conhecimento que tive em programação.

As matérias \textbf{Análise de Algoritmos} e \textbf{Introdução à Lógica e Verificação de Programas} me deram uma maior formalidade em análisar conceitos lógicos e algoritmos, me fazendo entender melhor otimização e estruturação de código. Foram de imensa ajuda para a produção de \textit{PsyChO: The Ball}.

As matérias \textbf{Programação Concorrente} e \textbf{Sistemas Operacionais} me deram o conhecimento de paralelismo e corotinas, o qual apliquei diretamente no sistema de leitura de \textit{scripts} pra níveis no \textbf{PsyChO: The Ball}.

As matérias: \textbf{Atividade Curricular em Cultura e Extensão}; \textbf{Design e Programação de Games}; \textbf{Laboratório de Programação II}; \textbf{Laboratório de Programação Extrema} e \textbf{Computação Móvel}, ofereceram atividades nas quais tive de fazer, na prática, um jogo (seja esse digital ou analógico). Em cada uma aprendi com os sucessos (e com os vários erros), podendo assim desenvolver e aprimorar técnicas de desenvolvimento de jogos. Quero fazer um agradecimento especial aos professores que me forneceram essas oportunidades de aplicar conhecimentos computacionais, em programação de jogos, algo que me incentivou a aprender e descobrir coisas novas.

Por último, o grupo extracurricular \textbf{UspGameDev} me ensinou muito mais do que eu poderia imaginar sobre desenvolvimento de jogos, \textit{game design}, \textit{game programming patterns} e me guiou durante todos projetos lúdicos extra-acadêmicos.

% ---------------------------------------------------------------------------- %
\section{Trabalhos Futuros}
\label{sec:trabalhos_futuros}

Meu objetivo é continuar trabalhando em \textit{PsyChO: The Ball} até chegar no estado desejado: 5 níveis, cada um com mecânicas e inimigos únicos e interessantes. É desejado lançar o jogo assim que possível em alguma plataforma digital de distribuição de jogos como a \textit{Steam}\footnote{store.steampowered.com}, \textit{Humble Bundle}\footnote{https://www.humblebundle.com/} ou \textit{Itch.io}\footnote{https://itch.io/}.

Além disso, quero continuar meus estudos nas área de Game Design e desenvolvimento de jogos, buscando sempre aprender novas técnicas, descobrir novas ferramentas e superar os diversos desafios que compõe essa área.
