% ---------------------------------------------------------------------------- %
\chapter{Introdução}
\label{cap:introducao}
% ---------------------------------------------------------------------------- %

  Desenvolvimento de jogos, ou \textit{game development}, é uma área de estudos interdisciplinar onde podemos aplicar conhecimentos de Arte, Música, \textit{Game Design} e, especialmente em jogos digitais, Ciência da Computação. Além disso, não é uma área na qual faltem desafios e problemas a serem solucionados. A produção de um jogo requer a aplicação direta de muitos conceitos e práticas ensinadas durante o curso em Ciência da Computação do Instituto de Matemática e Estatística da Universidade de São Paulo.

  Em novembro de 2009 foi fundado o \textbf{UspGameDev}, o grupo extracurricular da Universidade de São Paulo dedicado a fazer a ponte entre interessados em jogos e estudantes da faculdade, em especial estudantes da computação. Aberto à toda comunidade \textit{Uspiana}, a \textit{UspGameDev} ou \textit{UGD} sempre ficou de portas abertas para qualquer aluno que desejasse aprender mais sobre desenvolvimento de \textit{games}, tendo a chance de se reunir com outras pessoas e produzir seus próprios jogos, sejam eles digitais ou analógicos.

  Em 2013, no meu primeiro ano de faculdade, entrei na \textit{UGD} e, junto de um colega, fiz meu primeiro jogo virtual: \textit{PsyChObALL}, um \textit{top-down shooter psicodélico}. Foi o primeiro "grande" jogo que produzi durante a faculdade e isso me inspirou a continuar estudando na área de desenvolvimento de jogos, sempre conectado à \textit{UspGameDev}.

  Foi com essa mentalidade que decidi, no meio da graduação e com início na matéria \textit{Atividade Curricular em Cultura e Extensão}, produzir um \textit{remake} (termo análogo a uma "refilmagem", aplicado a jogos) de \textit{PsyChObALL} chamado de \textit{PsyChO: The Ball}, combinando todo o conhecimento adquirido durante os anos de estudos em computação e \textit{design} de jogos.


% ---------------------------------------------------------------------------- %
\section{Motivação e Objetivos}
\label{sec:motivacao_objetivo}

  O objetivo central desse trabalho final de formatura é aplicar todo o conhecimento aprendido durante a faculdade, focado na área de desenvolvimento de jogos (ou \textit{"game dev"}) e utilizar esse projeto como mais uma fonte de estudos e aprendizados. Neste sentido, me interessei em percorrer todas as etapas do desenvolvimento de um jogo digital.

  Primeiramente é necessária a discussão de \textit{balanço} sobre o jogo original \textit{PsyChObALL}, analisando seus pontos positivos e negativos. A segunda etapa seria construir uma base sólida para rodar o jogo, utilizando meus conhecimentos para criar bibliotecas e um ambiente apropriado para sua construção. A terceira etapa seria o desenvolvimento do jogo em si, utilizando como referência o jogo original. Por último, seria a etapa de testar com usuários e receber \textit{feedback}, podendo assim voltar à etapa 3 de desenvolvimento do jogo, melhorando o que for necessário, assim repetindo o ciclo.

  É esperado que depois de várias iterações, se chegue em um protótipo jogável de \textit{PsyChO: The Ball}, para que futuramente seja possível disponibilizar o jogo em alguma plataforma de distribuição de jogos digitais.
