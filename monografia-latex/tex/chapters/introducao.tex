%% ---------------------------------------------------------------------------- %
\chapter{Introdução}
\label{cap:introducao}
%% ---------------------------------------------------------------------------- %

Inteligência artificial, ou IA, é uma área de estudos que pode ser definida de diversas formas, como construir uma máquina que realize com sucesso tarefas tradicionalmente feitas por humanos, ou que aja como um humano.
Envolvendo filosofia, matemática, economia, neurociência, psicologia, computação e até mesmo linguística ao longo de sua história, ela abrangeu e ainda abrande diversos campos da ciência, com profissionais de várias formações diferentes podendo contribuir para seus avanços.
Existem inúmeros desafios, alguns resolvidos, como vencer de jogadores profissionais de Xadrez, mas muitos ainda sendo abordados, podendo ser uma busca por alguma solução, ou por uma solução mais eficiente que a que já existe.

Um tipo muito conhecido de inteligência artificial dos dias atuais é a que controla oponentes em jogos eletrônicos.
Porém, nesses casos, os adversários apenas seguem um conjunto pré-determinado de regras escritas pelo desenvolvedor, não possuindo a capacidade de se adaptar como seres humanos fazem.
Por mais que isso seja um tipo de IA e que possa ser mais eficiente em determinadas tarefas, não se assemelha à forma que as pessoas pensam e jogariam.
De forma geral, seres humanos aprendem interagindo com o ambiente: tocam nos objetos, tentam entender aquilo que os rodeia e qual o resultado de suas ações.
Em um jogo, se não passar por um tutorial ou ler um manual, não será muito diferente: o jogador precisará descobrir o que é hostíl, o que cada comando faz e qual o objetivo.
Avanços recentes em inteligência artificial permitem que máquinas simulem esse tipo de aprendizado, por meio da técnica \textbf{aprendizado por reforço}.

Entretanto, para a maior parte dos jogos eletrônicos, aprendizado por reforço não é o suficiente.
Em poucos movimentos, uma pessoa já consegue supor o que são inimigos e o que é terreno na tela do jogo.
Para um computador, um pixel que mude de posição já faz ele não conseguir mais distinguir o que está vendo.
Em outras palavras, seres humanos conseguem abstrair as informações que enxergam com facilidade, diferente de computadores.

%Antes de se chegar no estudo de inteligência artificial, já havia muitos questionamentos sobre como a mente, a inteligência funciona.
%Surgiram perguntas sobre o processo de raciocínio, se pode ser replicado por aparelhos mecânicos, se pode ser modelado matematicamente.
%Qual o papel das regras da lógica em sua operação, qual a influência das leis da física.
%Por muito tempo, houve questionamentos de natureza filosófica sobre a inteligência, que acabaram por influenciar o estudo de sua versão artificial.

%A transição dos questionamentos existentes para uma ciência formal requeriram formalização matemática em três áres fundamentais: lógica, computação, e probabilidade
%  Desenvolvimento de jogos, ou \textit{game development}, é uma área de estudos interdisciplinar onde podemos aplicar conhecimentos de Arte, Música, \textit{Game Design} e, especialmente em jogos digitais, Ciência da Computação. Além disso, não é uma área na qual faltem desafios e problemas a serem solucionados. A produção de um jogo requer a aplicação direta de muitos conceitos e práticas ensinadas durante o curso em Ciência da Computação do Instituto de Matemática e Estatística da Universidade de São Paulo.
%
%  Em novembro de 2009 foi fundado o \textbf{UspGameDev}, o grupo extracurricular da Universidade de São Paulo dedicado a fazer a ponte entre interessados em jogos e estudantes da faculdade, em especial estudantes da computação. Aberto à toda comunidade \textit{Uspiana}, a \textit{UspGameDev} ou \textit{UGD} sempre ficou de portas abertas para qualquer aluno que desejasse aprender mais sobre desenvolvimento de \textit{games}, tendo a chance de se reunir com outras pessoas e produzir seus próprios jogos, sejam eles digitais ou analógicos.
%
%  Em 2013, no meu primeiro ano de faculdade, entrei na \textit{UGD} e, junto de um colega, fiz meu primeiro jogo virtual: \textit{PsyChObALL}, um \textit{top-down shooter psicodélico}. Foi o primeiro "grande" jogo que produzi durante a faculdade e isso me inspirou a continuar estudando na área de desenvolvimento de jogos, sempre conectado à \textit{UspGameDev}.
%
%  Foi com essa mentalidade que decidi, no meio da graduação e com início na matéria \textit{Atividade Curricular em Cultura e Extensão}, produzir um \textit{remake} (termo análogo a uma "refilmagem", aplicado a jogos) de \textit{PsyChObALL} chamado de \textit{PsyChO: The Ball}, combinando todo o conhecimento adquirido durante os anos de estudos em computação e \textit{design} de jogos.
%
%
%% ---------------------------------------------------------------------------- %
%\section{Motivação e Objetivos}
%\label{sec:motivacao_objetivo}
%
%  O objetivo central desse trabalho final de formatura é aplicar todo o conhecimento aprendido durante a faculdade, focado na área de desenvolvimento de jogos (ou \textit{"game dev"}) e utilizar esse projeto como mais uma fonte de estudos e aprendizados. Neste sentido, me interessei em percorrer todas as etapas do desenvolvimento de um jogo digital.
%
%  Primeiramente é necessária a discussão de \textit{balanço} sobre o jogo original \textit{PsyChObALL}, analisando seus pontos positivos e negativos. A segunda etapa seria construir uma base sólida para rodar o jogo, utilizando meus conhecimentos para criar bibliotecas e um ambiente apropriado para sua construção. A terceira etapa seria o desenvolvimento do jogo em si, utilizando como referência o jogo original. Por último, seria a etapa de testar com usuários e receber \textit{feedback}, podendo assim voltar à etapa 3 de desenvolvimento do jogo, melhorando o que for necessário, assim repetindo o ciclo.
%
%  É esperado que depois de várias iterações, se chegue em um protótipo jogável de \textit{PsyChO: The Ball}, para que futuramente seja possível disponibilizar o jogo em alguma plataforma de distribuição de jogos digitais.
