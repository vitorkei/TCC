% https://en.wikipedia.org/wiki/Asteroids_(video_game)
% https://en.wikipedia.org/wiki/Golden_age_of_arcade_video_games
% https://www.arcade-museum.com/game_detail.php?game_id=6939
% https://www.ranker.com/list/the-most-popular-golden-age-arcade-games/video-games-lists
% https://stella-emu.github.io/index.html
% https://stella-emu.github.io/docs/index.html
%

% https://github.com/openai/retro
% https://blog.openai.com/gym-retro/
% https://openai.com/
% https://gym.openai.com/

% https://en.wikipedia.org/wiki/TensorFlow
% https://github.com/tensorflow/tensorflow
% https://www.tensorflow.org/
% RUSSEL, Stuart Jonathan and NORVIG, Peter - Artificial Intelligence: a mordern approach

% labels:
% cap:fundamentos
% sec:nn ............ neural network
% sec:dl ............ deep learning
% sec:cnn ........... convolutional neural network
% sec:mdp ........... markov decision process
% sec:rl ............ reinforcement learning
% sec:ql ............ q learning
% sec:aql ........... approximate q learning
% sec:dql ........... deep q learning
% sec:enhance ....... enhancements
% sec:er ............ experience replay
% sec:fx ............ fixed target
% sec:ddql .......... double deep q learning

%% ---------------------------------------------------------------------------- %
\chapter{Fundamentos}
\label{cap:fundamentos}
%% ---------------------------------------------------------------------------- %

Para se criar e treinar uma inteligência artificial, diversos arcabouçous são necessários.
Por um lado, existe a parte teórica e matemática na qual a inteligência se baseia para aprender. 
Por outro, do lado computacional, existem as bibliotecas que auxiliam no desenvolvimento, efetuando as contas necessárias e, neste trabalho em particular, emulando o jogo que serve de ambiente para o aprendizado.
Este capítulo tem o intuito de familiarizar o leitor com a teoria e fundamentos técnicas utilizados na modelagem e treinamento da inteligência artificial deste trabalho.

% ---------------------------------------------------------------------------- %

\section{Redes neurais}
\label{sec:nn}
%https://www.doc.ic.ac.uk/~nd/surprise_96/journal/vol4/cs11/report.html#What%20is%20a%20Neural%20Network
%https://www.youtube.com/watch?v=aircAruvnKk&list=WL
%https://towardsdatascience.com/the-mostly-complete-chart-of-neural-networks-explained-3fb6f2367464

Redes neurais artificiais, mais conhecidas como redes neurais, são uma forma de processamento de informação inspirada no funcionamento do cérebro.
Assim como o órgão no qual foi baseada, elas possuem uma grande quantidade de elementos de processamento de informação conectados entre si chamados de neurônios, que trabalham em conjunto para resolver problemas.
Dado que aprendem com exemplos, similar a pessoas, é considerada uma técnica de aprendizado supervisionado.

Com os avanços nos estudos dessa técnica nos últimos anos, diversos tipos diferentes de redes neurais foram desenvolvidos, como redes neurais convolucionais (\textit{Convolutional Neural Networks}, CNN), a utilizada neste trabalho, e redes neurais de memória de curto-longo prazo\footnote{Tradução livre feita pelo autor} (\textit{Long/Short Term Memory}, LSTM), que não será abordada.
Apesar de cada uma ter sua particularidade, redes neurais clássicas possuem duas características principais: os \textbf{neurônios}, e a estrutura dividida em \textbf{camadas}.
Existem redes que não são consideradas clássicas pela falta de estrutura em camadas, como Redes de Hopfield (\textit{Hopfield Network}) e Máquinas de Boltzmann (\textit{Boltzmann Machine}), que não serão discutidas neste trabalho.

\textbf{Neurônios} são funções que recebem como entrada a saída de cada neurônio da camada anterior, e devolvem um número, em geral entre 0 e 1 inclusos, cujo significado e como são usados variam de acordo com o trabalho em questão.

A estrutura de uma rede neural clássica é dividida em \textbf{camadas} que podem ser classificadas de três formas distintas: \textbf{entrada}, \textbf{oculta}, ou \textbf{saída}.
A \textbf{entrada} é o que a IA recebe inicialmente e precisa processar; as camadas \textbf{ocultas} (\textit{hidden layers}) são o processamento; e a \textbf{saída} é uma série de números utilizados pela IA para tomar uma decisão ou fazer uma predição.
Pode-se dizer que uma rede neural é um aproximador de uma função que mapeia entrada e saída.
Enquanto o número de neurônios na entrada e na saída são definidos pelo trabalho em questão, como número de pixels da tela e número de ações possíveis neste trabalho, o número de camadas ocultas e de neurônios em cada uma delas são arbitrários, sendo normalmente definidos por meio de tentativa e erro.

Cada neurônio das camadas ocultas representa uma característica (\textit{feature}) detectada ao longo do treinamento.
Se essa característica estiver presente na camada de entrada, então o neurônio correspondente a essa característica será \textbf{ativado}.
A ativação de um ou mais neurônios pode levar a ativação de outros neurônios na camada seguinte e assim sucessivamente. Esse é um comportamento inspirado na forma como neurônios do cérebro enviam sinais de um para o outro.
Em redes neurais artificiais, um neurônio é ativado quando a soma dos números de entrada passa de um certo valor e por uma função de ativação.

Porém, nem todos os valores de entrada devem ser igualmente importantes, então cada um desses números recebe um peso que determina sua importância para a ativação da característica.
Matematicamente, isso é representado da seguinte forma:
seja $n$ o número de neurônios na camada anterior, $w_{i}$, $i = 1, ..., n$, os pesos das saídas de cada neurônio da camada anterior, e $a_{i}$, $i = 1, ..., n$, o valor de saída de cada neurônio da camada anterior e, por consequência, cada valor de entrada do neurônio atual, e $b$ o viés (\textit{bias}) da função, que será explicado nos próximos parágrafos.

\begin{equation} \label{eq:nn01}
w_{1}a_{1} + w_{2}a_{2} + ... + w_{n}a_{n} - b
\end{equation}

Como essa soma pode ter qualquer valor no intervalo $(-\infty, +\infty)$, o neurônio precisa saber a partir de que ponto ele estará ativado.
Para isso, utiliza-se uma \textbf{função de ativação}.
Funções de ativação recebem a soma \ref{eq:nn01} como entrada, limitam seu valor a um certo intervalo 
e determinam se o neurônio deve ser ativado ou não.

Esse procedimento é feito em cada neurônio de cada camada da rede neural, o que pode ser muito custoso se não executado com cuidado.
Como existem diversas bibliotecas que otimizam operações matriciais, é mais rápido e conveniente utilizar matrizes, além de facilitar a leitura do código:
seja $W$ a matriz tal que cada linha contém os pesos de cada neurônio da camada anterior para um determinado neurônio da camada atual, $a^{(i)}$ o vetor tal que cada elemento é o valor de saída de cada neurônio da camada anterior, e $b$ o viés, é possível efetuar a soma \ref{eq:nn01} para todos os neurônios de uma camada da seguinte forma:

\begin{equation} \label{eq:nn02}
Wa^{(i)} + b, \qquad i = 1, ..., n
\end{equation}

As funções de ativação mais comuns são a sigmoide (curva logística), ReLU (\textit{Rectified Linear Unit}) e ELU (\textit{Exponential Linear Unit}), sendo a sigmoide a mais antiga e a ELU a mais recente.

O próximo passo é entender como os valores dos neurônios e os respectivos pesos são utilizados para a inteligência conseguir soltar a resposta correta.
Como mencionado anteriormente, conforme as características se mostram presentes na camada de entrada, os neurônios referentes a esses atributos são ativados, até que o neurônio com a resposta dada pela inteligência artificial seja ativado.
Como rede neural é um tipo de aprendizado supervisionado, os exemplos inseridos nela possuem rótulos, saídas esperadas (qual valor que cada neurônio de saída deve ter).
Para que o computador saiba o quão ruim foi sua saída, é definida uma função de erro, também conhecida como função de custo.
Em muitos casos, utiliza-se o erro quadrático médio, que dá mais peso para erros maiores uma vez que a diferença é elevada ao quadrado.
Naturalmente, quanto maior for o erro, mais incorreta foi a previsão.
Depois de esse procedimento ser feito com milhares de exemplos, calcula-se a média dos erros obtidos e, com isso, avalia-se o desempenho da inteligência.

Otimizar os pesos de forma que se reduz a média dos erros obtidos com os exemplos parece ser o melhor caminho para melhorar o modelo, mas isso não é necessariamente verdade.
Os milhares de exemplos utilizados nessa etapa compõe o conjunto de treinamento.
Se o modelo tiver erro zero em relação a esse conjunto, ele estará sofrendo de \textit{overfitting}: a inteligência se adequa tanto ao conjunto de treinamento que saberá o que fazer apenas nele.
O mais desejável é minimizar o erro sobre todos os dados possíveis ou, no mínimo, os esperados em cenários reais, e o conjunto de treinamento dificilmente conseguirá abranger todos eles.

De forma resumida, uma rede neural clássica aprende recebendo uma série de números como entrada e devolve uma saída;
calcula-se o quão errada essa saída está em relação ao desejado para aquela determinada entrada, e ajusta os pesos conforme a necessidade para minimizar o erro;
após repetir esses passos milhares de vezes, espera-se que a IA tenha aprendido o suficiente a resolver o problema em mãos.

% ---------------------------------------------------------------------------- %

\section{Aprendizado profundo}
\label{sec:dl}
%http://www.deeplearningbook.org/
%https://www.cs.toronto.edu/~vmnih/docs/dqn.pdf
%https://machinelearningmastery.com/what-is-deep-learning/
%https://stats.stackexchange.com/questions/182734/what-is-the-difference-between-a-neural-network-and-a-deep-neural-network-and-w

Como explicado anteriormente, redes neurais podem ser divididas em três tipos distintos de camadas: entrada, ocultas, e saída. Enquanto existe apenas uma camada de entrada e uma de saída, é possível haver uma ou mais camadas ocultas. Se houver muitas camadas ocultas, a rede neural passa a ser chamada de rede neural profunda (\textit{deep neural network}). Atualmente, não existe uma definição exata de quantas camadas a rede neural precisa ter para começar a ser classificada como profunda e, mesmo que houvesse, esse número provavelmente mudaria com o passar do tempo.

Uma rede neural profunda que segue o modelo apresentado na seção anterior é chamada de \textit{feedforward} e é o mais típico de \textit{deep learning}. Ele recebe esse nome pois a informação flui da entrada para a saída sem haver conexões de \textit{feedback} para que a previsão seja feita. Este tipo de rede neural forma a base para \textbf{redes neurais convolucionais}, técnica muito utilizada em reconhecimento de imagens. Como a ideia é treinar uma inteligência artificial que aprende vendo a tela do jogo, esse foi o tipo escohido para este trabalho.

% ---------------------------------------------------------------------------- %

\section{Rede neural convolucional}
\label{sec:cnn}
%https://medium.com/@ageitgey/machine-learning-is-fun-part-3-deep-learning-and-convolutional-neural-networks-f40359318721

Como neste trabalho a IA precisa aprender o que é um asteróide apenas enxergando a tela e interagindo com o ambiente, utilizar apenas redes neurais profundas sofre de um problema:
o computador não consegue reconhecer um mesmo objeto em diferentes locais da tela e de diferentes tamanhos como o mesmo.
Para cada local muito diferente que ele aparecer, como direita e esquerda da tela, a IA teria que re-aprender a identificá-lo.

Para não precisar fornecer à rede neural imagens inteiras para que ela aprenda que um objeto continua sendo o mesmo não importa onde da tela apareça, foi utilizada convolução, mais precisamente a 2D, já que a entrada é uma imagem, uma matriz de pixels.
Uma rede neural convolucional continua sendo um tipo de rede neural profunda e, portanto, mantém o formato de três tipos de camadas (entrada, ocultas, e saída). Porém, para facilitar o entendimento de convolução, esta explicação dividirá a rede em duas partes: \textbf{convolução} e \textbf{previsão}.
Na etapa de \textbf{convolucão}, a imagem de entrada é dividida em várias imagens menores; elas podem ser adjacentes ou parcialmente sobrepostas, sendo o segundo caso mais comum.
Cada uma dessas imagens menores é chamada de \textbf{filtro convolucional}.
É possível dizer que se um filtro convolucional for arrastado para o lado, o local onde ele parar será o próximo filtro convolucional.
Esse arrasto é chamado de passo (\textit{stride}) e a distância que o filtro é arrastado é chamada de tamanho do passo.
Em seguida, cada uma dessas imagens menores é passada por uma rede neural menor, sendo processada normalmente.
As saídas dessas redes neurais menores são então passadas como entrada para a próxima etapa.
Na etapa de \textbf{previsão}, a informação passa por uma ou mais redes neurais maiores que farão a previsão. Para diferenciá-la das redes da etapa de convolução, as desta fase são chamadas de \textit{fully-connected}.

É possível haver mais de uma camada de convolução assim como pode haver mais de uma camada \textit{fully-connected}, e isso pode ser mais vantajoso. 
Quanto mais camadas houver, mais precisa será a predição.
Entretanto, não só o custo de tempo e espaço aumenta, como há um limite para o quão melhor será o desempenho da IA.
A partir de um certo ponto, a melhora se torna ínfima em comparação com o tempo desprendido e, portanto, deixa de ser benéfico colocar mais camadas.

%% ---------------------------------------------------------------------------- %

\section{Processo de Decisão de Markov}
\label{sec:mdp}

Antes de falar sobre aprendizado por reforço, é necessário explicar o que é um \textbf{Processo de Decisão de Markov} (\textit{Markov Decision Process} - MDP).
Um MDP padrão possui as seguintes propriedades:
a probabilidade de se chegar em um estado futuro $S'$ dado que a inteligência artificial, também conhecida como agente, se encontra no estado $S$ depende apenas da ação $A$ tomada nesse estado $S$, o que caracteriza a \textbf{propriedade Markoviana};
existe um modelo probabilístico que caracteriza essa transição, dado por $P(S'|S,A)$;
todos os estados do ambiente e todas as ações que o agente pode tomar em cada estado são conhecidas;
e a recompensa é imediatamente recebida após cada ação ser tomada.

As probabilidades de o agente tomar cada ação em um dado espaço são definidas por uma política $\pi$.
A qualidade de uma política é medida por sua \textbf{utilidade esperada}, e a política ótima é denotada por $\pi^{*}$.
Para calcular $\pi^{*}$, utiliza-se um algoritmo de iteração de valor, que computa a utilidade esperada do estado atual:
começando a partir de um estado arbitrário $S$, tal que seu valor esperado é $V(S)$, aplica-se a equação de Bellman até haver convergência de $V(S)$, que será denotado por $V^{*}(S)$.
Esse $V^{*}(S)$ é usado para calcular a política ótima $\pi^{*}(s)$.

Seja $i$ a iteração atual, $S$ o estado atual, $S'$ o estado futuro, $A$ a ação tomada no estado atual, $R(S,A,S')$ a recompensa pela transição do estado $S$ para o estado $S'$ por tomar a ação $A$, e $\gamma$ o valor de desconto (valor entre 0 e 1 que determina a importância de recompensas futuras para o agente), temos que:

Equação de Bellman:

\begin{equation} \label{eq:bellman}
V^{(i)}(S) = \max_{A}\sum_{S'}P(S'|S,A)[R(S,A,S') + \gamma V^{(i-1)}(S')]
\end{equation}

\begin{equation} \label{eq:qvalue}
\lim_{i\to\infty} V^{(i)}(S) = V^{*}(S)
\end{equation}

Política gulosa para função valor ótima:

\begin{equation} \label{eq:opt_pol}
\pi^{*}(s) = \argmax_{A}\sum_{S'}P(S'|S,A)[R(S,A,S') + \gamma V^{*}(S')]
\end{equation}

Um dado das fórmulas acima comum de se faltar é a probabilidade de transição $P(S'|S,A)$ e isso não é diferente no jogo \textit{Asteroids}.
Portanto, utiliza-se \textbf{aprendizado por reforço} para contornar esse problema.

%% ---------------------------------------------------------------------------- %

\section{Aprendizado por reforço}
\label{sec:rl}

Aprendizado por reforço, diferente do supervisionado e, por consequência, de redes neurais, não recebe exemplos rotulados para saber o quão incorreta sua resposta está para cada entrada.
Ao invés disso, o agente interage com o ambiente e recebe recompensas positivas, negativas ou nulas por suas ações.
Seu objetivo é explorar o espaço de estados a fim de aprender a recompensa esperada para cada ação tomada em cada um deles.
Dessa forma, ele saberá o que fazer em cada situação do ambiente em que se encontra.

As recompensas esperadas de cada ação em cada estado são armazenadas em uma tabela que deve mapear todas as ações para todos os estados.
Isso é possível em domínios simples, como um Jogo da Velha ou um \textit{Gridworld}, mas se torna impraticável conforme o espao de estados aumenta.
No caso do jogo \textit{Asteroids}, os \textit{frames} do jogo são os estados.
Um pixel que mude de cor já faz ser um estado completamente diferente do ponto de vista do computador.
Em uma tela de 210x160 pixels, com cada pixel armazenando três números que vão de 0 à 255 para determinar sua cor, é evidente não ser possível armazenar na memória um mapeamento das ações para cada um desses estados.
Mesmo que não houvesse esse obstáculo computacional, há muitas situações em que não é possível determinar qual ação retornará a maior recompensa.

Como dito no final da seção \hyperref[sec:mdp]{anterior}, aprendizado por reforço é um MDP que não utiliza as probabilidades de transição para aproximar a política ótima.
No contexto deste trabalho, a política ótima será encontrada utilizando uma variante da técnica \textit{\textbf{Q-Learing}}.

%% ---------------------------------------------------------------------------- %

\section{\textit{Q-learning}}
\label{sec:ql}

Quando não se conhece as probabilidades de transição, informação necessária para se obter a função valor pela equação de Bellman, é possível estimar $V(S)$ a partir de observações feitas sobre o ambiente.
Logo, o problema deixa de ser tentar encontrar $P$ e passa a ser como extrair a política do agente de uma função valor estimada.

Seja $Q^{*}(S,A)$ a função Q-valor\footnote{O nome "Q-valor"{} vem do valor da qualidade da ação} que expressa a recompensa esperada de se começar no estado $S$, tomar a ação $A$ e continuar de maneira ótima. $Q^{*}(S,A)$ é uma parte da política gulosa para função valor ótima e é dada por:

\begin{equation} \label{eq:qfunction}
\begin{align*}
Q^{*}(S,A) &= \sum_{S'}P(S'|S,A)[R(S,A,S') + \gamma V^{*}(S')] \\
        &= \sum_{S'}P(S'|S,A)[R(S,A,S') + \gamma \max_{A'}Q^{*}(S',A')]
\end{align*}
\end{equation}

Logo, substituindo \ref{eq:qfunction} em \ref{eq:opt_pol}, temos que a política gulosa ótima para a função Q-valor ótima é dada por:

\begin{equation} \label{eq:q_opt_pol}
\pi^{*}(S) = \argmax_{A}Q^{*}(S,A)
\end{equation}

O próximo passo será entender como atualizar a função Q-valor.

Supondo que o agente se encontra no estado $S$ e toma a ação $A$, que causa uma transição no ambiente para o estado $S'$ e gera uma recompensa $R(S,A,S')$, como computar $Q^{(i+1)}(S,A)$ baseado em $Q^{(i)}(S,A)$ e em $R(S,A,S')$, sendo $i$ o momento atual?
Para responder a essa pergunta, duas restrições precisam ser feitas: $Q^{(i+1)}(S,A)$ deve obedecer, pelo menos de forma aproximada, a equação de Bellman, e não deve ser muito diferente de $Q^{(i)}(S,A)$, dado que são médias de recompensas.
A seguinte equação responde a essa questão.

Seja $\alpha$ a taxa de aprendizado (valor entre 0 e 1 que determina o quão importantes informações novas são em relação ao conhecimento que o agente possui),

\begin{equation} \label{eq:q_update}
\begin{align*}
Q^{(i+1)}(S,A) &= (1-\alpha)Q^{(i)}(S,A) + \alpha[R(S,A,S') + \gamma \max_{A'}Q^{(i)}(S',A')] \\
            &= Q^{(i)}(S,A) + \alpha[R(S,A,S') + \gamma \max_{A'}Q^{(i)}(S',A') - Q^{(i)}(S,A)]
\end{align*}
\end{equation}

A convergência de $Q^{(i)}(S,A)$ em $Q^{*}(S,A)$ é garantida mesmo que o agente aja de forma subótima contanto que o ambiente seja um MDP, a taxa de aprendizado seja manipulada corretamente, e se a exploração não ignorar alguns estados e ações por completo - ou seja, raramente.
Mesmo que as condições sejam satisfeitas, a convergência provavelmente será demasiadamente lenta.
Entretanto, é interessante analisar os problemas levantados pela segunda e pela terceira condição que garantem a convergência e maneiras de solucioná-los.

Se a \textbf{taxa de aprendizado} for muito alta (próxima de 1), a atualização do aprendizado se torna instável.
Por outro lado, se for muito baixa (próxima de 0), a convergência se torna lenta.
Uma solução possível para essa questão é utilizar valores que mudam de acordo com o estado: utilizar valores mais baixos em estados que já foram visitados muitas vezes, pois o agente já terá uma boa noção da qualidade de cada ação possível, então há pouco que aprender; e utilizar valores mais altos em estados que foram visitados poucas vezes, pois o agente precisa aprender melhor sobre o estado.

Uma vez que a política é gulosa em relação ao Q-valor, o agente sempre tomará a ação que retorna a maior recompensa esperada.
Ou seja, a ação escolhida depende do valor da taxa de desconto $\gamma$: recompensas imediatas serão mais buscadas se for próximo de 0, enquanto recompensas futuras serão mais valorizadas para valores próximos de 1.
Isso é bom somente se todas as recompensas possíveis para aquele estados são conhecidas.
Porém, se houver ações não exploradas, o agente pode perder uma recompensa maior do que as que ele já conhece apenas porque ignorou a ação que leva a ela.
Essa situação caracteriza o dilema \textbf{\textit{Exploration versus Exploitation}}: é melhor tomar a ação que retorna a maior recompensa ou buscar uma melhor?
Da mesma forma que na taxa de aprendizado, uma forma de contornar esse problema é mudar a probabilidade de decidir explorar o ambiente (\textit{explore}) de acordo com a situação.
Conforme o mundo é descoberto, se torna cada vez mais interessante agir de forma gulosa (\textit{exploit}) do que explorar em estados muito visitado, e vice-versa em estados pouco visitados.
Esse comportamento pode ser definido por uma função de exploração (\textit{exploration function}).

Seja $P_{ini}$ a probabilidade inicial e $P_{min}$ a probabilidade mínima de o agente decidir explorar (\textit{explore}) o ambiente, $decay$ a taxa de decaimento e $step$ o número de passos dados até o momento.
A probabilidade de o agente explorar (\textit{explore}) o ambiente é dada por:

\begin{equation} \label{eq:exp_exp_prob}
%P_{explore} = P_{min} + (P_{ini} - P_{min})e^{decay * step}
P_{explore} =
\begin{cases}
P_{ini} - decay * step, & \text{se}\ decay * step <= 0.9 \\
P_{min}, & \text{caso contrário}
\end{cases}
\end{equation}

Outro problema enfrentado por \textit{Q-learning} é o de generalização.
A política $\pi^{*}(S)$ determina a melhor ação a se tomar em cada estado.
Logo, utiliza-se uma tabela para armazenar todas essas escolhas.
Porém, como mencionado anteriormente, isso se torna inviável para espaços de estado muito grandes.
Portanto, se não é possível aprender os Q-valores, o melhor que se pode fazer é tentar achar uma aproximação deles.
%Portanto, a solução é generalizar o aprendizado de um estado para o outro: se o agente sabe se comportar em um pequeno conjunto de estados, o ideal é ele saber o que fazer em um estado desconhecido contanto que seja suficientemente parecido com um já aprendido.
%Em outras palavras, o agente aprende propriedades (\textit{features}) dos estados ao invés dos estados propriamente ditos, e toma decisões a partir dessas informações.
%Essa forma de aprender a fazer escolhar é chamada de \textit{\textbf{approximate Q-learning}}.

%% ---------------------------------------------------------------------------- %

\section{\textit{Approximate Q-learning}}
\label{sec:aql}

\textit{Approximate Q-Learning} é o nome dado a um conjunto de métodos de aprendizado por reforço que busca aproximar o Q-valor das ações.
Uma técnica comum desse conjunto é o uso de \textbf{funções lineares} que avaliam características dos estados.

Para lidar com o enorme espaço de estados que alguns ambientes possuem, o agente armazena e aprende apenas algumas propriedades, que são funções de valor real, para tomar as decisões.
Tais informações são armazenadas em um vetor e cada elemento desse vetor recebe um peso que determina a respectiva importância para que escolhas sejam feitas. Ou seja, a função Q-valor é representada por uma combinação linear das propriedades e é dada da seguinte forma:

\begin{equation} \label{eq:q_lin_comb}
Q(S,A) = \omega_{1}f_{1}(S,A) + \omega_{2}f_{2}(S,A) + ... + \omega_{n}f_{n}(S,A)
\end{equation}

Como o $V(S')$ é o valor esperado e $Q(S,A)$ é o valor previsto, a atualização pode ser interpretada como ajustar o Q-valor pela diferença desses dois número. Além disso, como o \textit{approximate Q-learning} avalia características, apenas os pesos precisam ser atualizados:

\begin{equation} \label{eq:w_update}
\omega_{k}^{(i+1)}(S,A) = \omega_{k}^{(i)}(S,A) + \alpha[R(S,A,S') + \gamma V(S') - Q^{(i)}(S,A)]f_{k}(S,A), \hfill k = 1, 2, ..., n
\end{equation}

Duas grandes vantagens de representar o Q-valor como uma combinação linear são evitar \textit{overfitting} (a IA aprender tanto com o conjunto de treinamento que não consegue tomar decisões que diferem demais dele), e ser matematicamente conveniente, ter maneiras convenientes de calcular erro e funções que generalizem as decisões.

Percebe-se neste ponto uma semelhança bem grande com a forma como redes neurais profundas funcionam.

%% ---------------------------------------------------------------------------- %

\section{\textit{Deep Q-Learning}}
\label{sec:dql}

Agora que as técnicas aprendizado profundo e aprendizado por reforço foram apresentadas, falta falar do tipo de aprendizado obtido quando é feita a junção delas, que é o utilizado neste trabalho: \textit{\textbf{Deep Q-Learning}}.
Por se tratar de um tipo de aprendizado por reforço, mais precisamente \textit{approximate Q-learning}, a inteligência artificial também será referida como agente.

Como a ideia é o agente aprender enxergando a tela e tudo que ele vê são matrizes, a primeira etapa consiste em processar essas informações para poder aprender.
Ou seja, o primeiro passo é passar os \textit{frames} da tela do jogo por uma \textbf{rede neural convolucional}.
Essa etapa funciona como descrito na seção sobre \hyperref[sec:cnn]{redes neurais convolucionais}: os \textit{frames} são a entrada, ocorre o processamento nas camadas ocultas e, na camada de saída, cada neurônio tem um valor.
Contudo, calcular o erro é diferente.
Como as imagens passadas não são rotuladas e não é possível fazer isso manualmente, a IA precisa calcular o erro da saída da rede de outra forma.
É nessa etapa que entra o \textbf{aprendizado por reforço}.
Como explicado \hyperref[sec:rl]{anteriormente}, aprendizado por reforço aprende sem o uso de exemplos rotulados, mas precisa que as características que a IA deve aprender estejam bem definidas.
Enquanto tais características são definidas pela rede, o cálculo do erro e otimização dos pesos é feito pela parte de aprendizado por reforço.

%% ---------------------------------------------------------------------------- %

\section{Aprimorando o aprendizado}
\label{sec:enhance}

\textit{Deep Q-Learning} é a base para o aprendizado da inteligência artificial deste trabalho.
Porém, ao longo do tempo, foram descobertas outras técnicas que melhoram a aprendizagem do agente, seja acelerando ou evitando decisões nocivas.
Nesta seção, serão apresentadas
%três
duas técnicas que ajudam a estabilizar e acelerar o aprendizado do programa:
\textit{\textbf{experience replay}}
%, alvo fixo
e \textit{\textbf{double} deep Q learning}.

%***********%

\subsection{\textit{Experience Replay}}
\label{sec:er}

Por conta da forma como \textit{Deep Q-Learning} funciona, se a rede aprendesse com os \textit{frames} conforme o agente os vê, a entrada seria composta de estados sequenciais.
Isso significa que ela se atualizaria apenas com as experiências passadas mais próximas, "esquecendo"{} as mais antigas.
Para contornar esse problema, utiliza-se um \textit{buffer} que armazena experiências anteriores e, a cada ação feita, ao invés de passar a transição de estados resultado dessa ação, passa-se uma amostra de experiências escolhidas aleatoriamente desse \textit{buffer} para a rede ser atualizada.
Utilizar \textit{experience replay} também ajuda o agente a tomar ações diferentes ao invés de se prender a algumas que tiveram sucesso no passado próximo e que, possivelmente, não serão tão úteis no futuro.

%***********%

\subsection{Alvo fixo}
\label{sec:ft}

A inteligência artificial utiliza os Q-valores tanto para decidir quais ações tomar quanto para atualizá-los na tentativa de melhor aproximar o Q-valor real.
Entretanto, se o mesmo for utilizado para essas duas etapas, o aprendizado se torna instável, pois o agente não conseguirá "alcançar"{} o Q-valor alvo já que ele está em constante mudança.
Para resolver esse problema, utiliza-se uma segunda rede neural alvo para a atualização dos Q-valores e que é atualizada de tempos em tempos, enquanto a primeira, a principal, continua sendo utilizada para escolher as ações.

Essa segunda rede também é utilizada com \textit{\textbf{Double Deep Q-Learning}}.

%***********%

\subsection{\textit{Double Deep Q-Learning}}
\label{sec:ddql}

\textit{Double Deep Q-Learning} é uma aplicação de \textit{double Q-learning} para \textit{deep Q-learning}.
Esta técnica utiliza duas redes neurais: uma que escolhe a melhor ação a ser tomada e uma que calcula o Q-valor dessa ação, a mesma da técnica \hyperref[sec:ft]{alvo fixo}.

Como o Q-valor é uma avaliação da qualidade de uma ação, se ações sub-ótimas frequentemente receberem Q-valores maiores que a ação ótima, o agente terá dificuldades em saber qual a melhor decisão a se tomar em cada instante.
Isso ocorre em particular no começo do treinamento, quando não se tem informação suficiente sobre as ações e existe ruído, tornando o aprendizado instável e lento.

A solução proposta por \textit{double deep Q-learning} é separar a escolha da ação $A$ no estado $S$ (feita por uma rede QN) do cálculo do Q-valor alvo (feita por uma rede TN).
Lembrando da forma como se \hyperref[eq:q_update]{calcula Q-valor}, a alteração será no fator $\max_{A'}Q^{(i)}(S',A')$, que é trocado por $Q_{TN}(S', \argmax_{A'}Q_{QN}(S',A'))$.
Fazendo essa substituição em \ref{eq:q_update} e identificando as funções $Q()$ devidamente, chega-se à seguinte forma de se calcular o Q-valor:

\begin{equation} \label{d_q_update}
Q_{QN}^{(i+1)}(S,A) = Q_{QN}^{(i)}(S,A) + \alpha[R(S,A,S') + \gamma Q_{TN}^{(i)}(S',\argmax_{A'}Q_{QN}^{(i)}(S',A')) - Q_{QN}^{(i)}(S,A)]
\end{equation}

A atualização dos Q-valores da rede TN é feita de maneira simétrica, com $Q_{TN}$ calculando a melhor ação e $Q_{QN}$ calculando o respectivo Q-valor.

%Lembrando que o Q-valor é a recompensa esperada de se tomar uma ação $A$ em um estado $S$. Como isso depende da recompensa esperada nos estados futuros, 
%A precisão do Q-valor depende de quais ações $A$ foram tomadas e quais estados vizinhos $S'$ foram explorados para um dado estado $S$.
%No início do aprendizado, não se tem informação suficiente sobre qual a melhor ação a ser tomada, o que significa que a que tiver maior Q-valor provavelmente não será a ação ótima.

%***********%

%\subsection{\textit{Dueling Deep Q-Learning}}
%\label{sec:dueling}
